\section*{resumen}

Con el tiempo, el uso de artículos wiki basados en MediaWiki,
como Wikipedia, para la búsqueda de información, se ha incrementado de
manera significativa.
Existen aplicaciones y herramientas que hacen uso de estos artículos
para analizar su contenido con diversos propósitos.
Tal es el caso de Wiki-Metrics-UCV, cuya labor es la extracción,
almacenamiento y análisis de las revisiones de dichos artículos.
El almacenamiento de las revisiones de artículos wiki,
puede llegar a ser una tarea compleja debido a la magnitud de los datos que se maneja y las limitaciones de hardware y espacio de almacenamiento que una máquina pueda tener.
Por esta razón, se decidió crear una nueva aplicación que incorpora
nuevas tecnologías que permiten escalar de manera horizontal, la capacidad
de almacenamiento de datos a través de múltiples máquinas trabajando en conjunto, y de esta manera,
superar las limitaciones de hardware que una sola máquina pueda presentar.
Este trabajo de investigación se centra en la implementación de una aplicación distribuida,
la cual permite la extracción de las revisiones de artículos wiki, su almacenamiento y la
construcción de diversas consultas que permiten obtener métricas asociadas al historial
de modificación de un artículo.
Esta aplicación distribuida se desarrolló por medio de la aplicación de la metodología
ágil llamada Desarrollo Rápido de Aplicaciones (RAD), la cual se basa en el desarrollo
continuo de múltiples componentes pequeños que en conjunto conforman todo el sistema.

\textbf{Palabras clave:} Aplicación distribuida, Base de Datos, Nodos, API, Servicio Web, Algoritmo, Cluster, Consultas.
