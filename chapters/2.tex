\section{Tecnologías}

\subsection{Mongo}

MongoDB es una base de datos NoSQL (no solo SQL) orientada a documentos. Almacena
datos en una representación binaria llamada BSON (Binary JSON), la cual extiende
las capacidades de un objeto JSON (JavaScript Object Notation) para representar
tipos de datos como enteros, punto flotante, fechas, datos binarios, arreglos y sub-documentos o
documentos embebidos.

MongoDB posee un conjunto de  características que permiten la escalabilidad de una aplicación, entre ellas,
provee la habilidad e implementar la distribución geográfica de datos (sharding). Esto permite a la
base de datos ser escalada de forma horizontal con el uso de un conjunto de componentes de hardware o mediante
de la nube (cloud).

Adicionalmente, MongoDB esta diseñado para ser ejecutado en un sistema de múltiples nodos, por lo tanto,
en la presencia de un escalamiento horizontal, incluye la capacidad para la replicación
y sincronización de datos entre todos los componentes del sistema. De esta manera se garantiza la posibilidad
de implementar un servicio con alta disponibilidad \cite{10}.

\subsubsection{Replicación}

MongoDB maintains multiple copies of data called replica
sets using native replication. A replica set is a fully
self-healing shard that helps prevent database downtime
and can be used to scale read operations. Replica failover
is fully automated, eliminating the need for administrators
to intervene manually.
A replica set consists of multiple replicas. At any given time
one member acts as the primary replica set member and
the other members act as secondary replica set members.
MongoDB is strongly consistent by default: reads and
writes are issued to a primary copy of the data. If the
primary member fails for any reason (e.g., hardware failure,
network partition) one of the secondary members is
automatically elected to primary, typically within several
seconds. As discussed below, sophisticated rules govern
which secondary replicas are evaluated for promotion to
the primary member


\subsubsection{Sharding}


\subsection{Celery}

\subsection{RabbitMQ}

\subsection{Flask}

\subsubsection{Flask AutoDoc}

% Is this subsection necessary?
\subsection{Wikipedia API}

