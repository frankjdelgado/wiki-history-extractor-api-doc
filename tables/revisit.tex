\begin{longtable}{|l|m{4in}|}

\hline
\multicolumn{2}{|r|}{\textbf{Historia}} \\
\hline
\endfirsthead

\multicolumn{2}{c}%
{{\bfseries \tablename\ \thetable{} -- continuación de la página anterior}} \\
\hline \multicolumn{2}{|r|}{\textbf{Historia}} \\ \hline
\endhead

\textbf{Desarrollador} & Marvin Bernal \\
\hline
\textbf{Nombre} & Algoritmo de Revisita \\
\hline
\textbf{Sección} & Extracción \\
\hline
\textbf{Descripción} & Algoritmo para la extracción de nuevas revisiones de los artículos almacenados. Calcula la esperanza matemática de cada artículo en base a la función probabilística exponencial, 
para estimar cuando un artículo tenga una o mas revisiones pendientes no almacenadas en la base de datos. 
\par Para el cálculo de la esperanza se toma el 10\% más reciente de revisiones, 
y el tiempo transcurrido entre estas. 
Con estos datos, se calcula la esperanza y se verifica si el tiempo desde la última revisión es mayor a la misma. En caso positivo, el artículo es encolado para su extracción.
\\
\hline
\textbf{Observaciones} & Se ejecuta en segundo plano por medio de un cronjob diariamente.\\
\hline
\caption{Algoritmo de revisita de revisiones de artículos wiki}
\label{tab:revisit}
\end{longtable}
