\begin{longtable}{|l|m{4in}|}

\hline
\multicolumn{2}{|r|}{\textbf{Historia}} \\
\hline
\endfirsthead

\multicolumn{2}{c}%
{{\bfseries \tablename\ \thetable{} -- continuación de la página anterior}} \\
\hline \multicolumn{2}{|r|}{\textbf{Historia}} \\ \hline
\endhead

\textbf{Desarrollador} & Francisco Delgado \\
\hline
\textbf{Nombre} & Endpoint para la consultas directas a la base de datos \\
\hline
\textbf{Sección} & Consulta\\
\hline
\textbf{Descripción} & Parte del API de la solución distribuida donde el endpoint \texttt{/query}
recibe el contenido de la consulta a realizar en formato JSON, a través del cuerpo de la solicitud que se realiza. Dicho cuerpo debe tener un formato adecuado para realizar la consulta en base al módulo Pymongo.
\par Además de esto, existen 2 argumentos opcionales:
\par
\tabitem collection: la colección sobre la cuál se hará la consulta.
 Su valor por defecto es la colección \texttt{revisions}.
\par
\tabitem date\_format: el formato se fecha que será usado en la consulta. 
Su valor por defecto es \texttt{\%Y-\%m-\%dT\%H:\%M:\%S}. 
Donde cada letra excepto la T representa el valor de unidad de tiempo correspondiente en inglés, por ejemplo Y para Año (Year). 
La letra T representa la finalización de la sección de la fecha y el inicio de la sección del tiempo en el formato.

\\
\hline
\textbf{Observaciones} & \\
\hline
\caption{Consultas Directas a la Base de Datos}
\label{tab:query}
\end{longtable}
