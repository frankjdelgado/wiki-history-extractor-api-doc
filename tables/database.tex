\begin{longtable}{|l|m{4in}|}

\hline
\multicolumn{2}{|r|}{\textbf{Historia}} \\
\hline
\endfirsthead

\multicolumn{2}{c}%
{{\bfseries \tablename\ \thetable{} -- continuación de la página anterior}} \\
\hline \multicolumn{2}{|r|}{\textbf{Historia}} \\ \hline
\endhead

\textbf{Desarrollador} & Francisco Delgado \\
\hline
\textbf{Nombre} & Almacenamiento de Revisiones \\
\hline
\textbf{Sección} & Base de Datos \\
\hline
\textbf{Descripción} & Conjunto de funciones para la conexión, consulta e inserción de
datos en la base de datos (MongoDB). Realizada con el lenguaje
Python y definidas dentro de una clase, se encarga de brindar una
capa de interacción entre MongDB y todas las operaciones de
lectura/escritura de revisiones de un articulo wiki.
\\
\hline
\textbf{Observaciones} & Las operaciones de escritura detectan si la revisión a insertar es
un duplicado por medio del identificador único de la revisión
(revid), y posteriormente procede a actualizar los datos ya existentes con los de la nueva entrada.
\par
Se genera una nueva conexión con la base de datos por cada
operación que se requiera ejecutar en vez de utilizar una o más
conexiones en común para llevar a cabo las tareas.\\
\hline
\caption{Almacenamiento de revisiones de artículos wiki}
\label{tab:database}
\end{longtable}
