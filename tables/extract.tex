\begin{center}
\begin{longtable}{|l|m{4in}|}

\hline
\multicolumn{2}{|r|}{\textbf{Historia}} \\
\hline
\endfirsthead

\multicolumn{2}{c}%
{{\bfseries \tablename\ \thetable{} -- continuación de la página anterior}} \\
\hline \multicolumn{2}{|r|}{\textbf{Historia}} \\ \hline
\endhead

\textbf{Desarrollador} & Marvin Bernal \\
\hline
\textbf{Nombre} & Extracción de revisiones por API \\
\hline
\textbf{Sección} & Extracción\\
\hline
\textbf{Descripción} & Extracción de todas las revisiones de un artículo
	wiki dado su título a través del API que ofrece mediawiki \par
	Se realiza una petición sobre el URL
	\texttt{https://en.wikipedia.org/w/api.php} y se agregan parámetros
	extra, incluyendo el título del artículo, para obtener todos los
	metadatos posibles de una revisión.
	\par
	Cada petición al API obtiene un máximo de 50 revisiones se llevan
	a cabo con una diferencia de 2 segundos.
\\
\hline
\textbf{Observaciones} & Se extraen los siguientes datos: id de la revisión, tipo de revisión,
fecha, nombre de usuario, id de usuario, tamaño, comentarios,
contenido y etiquetas\\
\hline
\caption{Extracción de Revisiones}
\label{tab:extract_revisions}
\end{longtable}
\end{center}
