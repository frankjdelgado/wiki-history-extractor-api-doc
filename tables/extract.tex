\begin{longtable}{|l|m{4in}|}

\hline
\multicolumn{2}{|r|}{\textbf{Historia}} \\
\hline
\textbf{Sección} & Extracción\\
\hline
\endfirsthead

\multicolumn{2}{c}%
{{\bfseries \tablename\ \thetable{} -- continuación de la página anterior}} \\
\hline \multicolumn{2}{|r|}{\textbf{Historia}} \\ \hline
\textbf{Sección} & Extracción\\
\hline
\endhead
\textbf{Desarrollador} & Marvin Bernal, Francisco Delgado \\
\hline
\textbf{Nombre} & Extracción de revisiones por API \\
\hline
\textbf{Descripción} & Extracción de todas las revisiones de un artículo
	wiki dado a través del API que ofrece mediawiki. \par
	Se realiza una petición sobre el URL
	\texttt{https://en.wikipedia.org/w/api.php} para obtener todos los
	metadatos posibles de una revisión.
	\par
	Cada petición al API obtiene un máximo de 50 revisiones se llevan
	a cabo con una diferencia de 2 segundos.
\\
\hline
\textbf{Observaciones} & Se extraen los siguientes datos: id de la revisión, tipo de revisión,
fecha, nombre de usuario, id de usuario, tamaño, comentarios,
contenido y etiquetas. \par El título del artículo no es variable. \par El idioma(locale) del artículo no es variable. \\
\hline
\hline
\textbf{Desarrollador} & Marvin Bernal \\
\hline
\textbf{Nombre} & Extracción en caso de error del servidor \\
\hline
\textbf{Descripción} & Se modifica el método de extracción, para contemplar el escenario
donde el servidor presenta un error que lo inhabite o es desactivado manualmente.
\par Se añade un campo para tener la última revisión extraida, para que en el proceso de extracción se compare dicho campo con las revisiones extraidas, y poder seguir extrayendo a partir de dicha revisión. \\
\hline
\textbf{Observaciones} & Se añade el siguiente dato a la solicitud de extracción: rvstartid.\\

\hline
\hline
\textbf{Desarrollador} & Marvin Bernal \\
\hline
\textbf{Nombre} & Extracción en orden ascendente \\
\hline
\textbf{Descripción} & Se modifica el método de extracción, para realizar las extracciones de las mas antiguas a las mas nuevas, para facilitar el proceso de extracción y prevenir saltos en casos de errores o interrupciones si se crean nuevas revisiones durante el proceso. \\
\hline
\textbf{Observaciones} & \\

\hline
\hline
\textbf{Desarrollador} & Francisco Delgado \\
\hline
\textbf{Nombre} & Creación de artículos al extraer \\
\hline
\textbf{Descripción} & Al realizar las extracciones, se añade el artículo wiki a la colección de artículos de la base de datos. Si el artículo ya se encuentra, se modifica la ultima revisión extraida del artículo. \\
\hline
\textbf{Observaciones} & Añadida colección de artículos a la base de datos\\

\hline
\hline
\textbf{Desarrollador} & Marvin Bernal \\
\hline
\textbf{Nombre} & Extracción por título\\
\hline
\textbf{Descripción} & Se modifica el método de extracción para aceptar título del artículo a extraer, del cual se obtiene el pageid\\
\hline
\textbf{Observaciones} & Se añade el siguiente dato a la solicitud: título\\

\hline
\hline
\textbf{Desarrollador} & Francisco Delgado \\
\hline
\textbf{Nombre} & Incorporar extracción a tareas de Celery\\
\hline
\textbf{Descripción} & Se incorpora Celery a la extracción, para que le sea asignado un nodo trabajador a la tarea.\\
\hline
\textbf{Observaciones} & Añadido status a la tarea de extracción por celery, para poder seguir el estado de la tarea.\\


\hline
\hline
\textbf{Desarrollador} & Francisco Delgado \\
\hline
\textbf{Nombre} & Extracción en otros idiomas\\
\hline
\textbf{Descripción} & Se añade el campo local a la extracción, para permitir trabajar con artículos wiki en otros idiomas.\\
\hline
\textbf{Observaciones} & Añadido dato locale al proceso de extracción\\

\hline
\caption{Extracción de Revisiones}
\label{tab:extract}
\end{longtable}
	