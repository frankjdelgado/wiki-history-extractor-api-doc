\begin{longtable}{|l|m{4in}|}

\hline
\multicolumn{2}{|r|}{\textbf{Historia}} \\
\hline
\endfirsthead

\multicolumn{2}{c}%
{{\bfseries \tablename\ \thetable{} -- continuación de la página anterior}} \\
\hline \multicolumn{2}{|r|}{\textbf{Historia}} \\ \hline
\endhead

\textbf{Desarrollador} & Marvin Bernal \\
\hline
\textbf{Nombre} & Endpoint para el cálculo de la moda de las revisiones \\
\hline
\textbf{Sección} & Consulta\\
\hline
\textbf{Descripción} & Parte del API de la solución distribuida donde el endpoint \texttt{/mode}
recibe como parámetros los atributos para restringir o filtrar el
rango a tomar en cuenta como resultado, adicionalmente se usa el parámetro \texttt{attribute} para señalar el atributo sobre el cual calcular la moda, que consiste en el valor con mayor número de repeticiones entre el conjunto de valores obtenidos.  
Entre los valores posibles que puede asumir \texttt{attribute} se encuentran:
\par
\tabitem title: el título del artículo con mayor cantidad de revisiones.
\par
\tabitem pageid: el id del artículo con mayor cantidad de revisiones.
\par
\tabitem user: nombre del autor que mas revisiones haya realizado.
\par
\tabitem userid: id del autor que mas revisiones haya realizado.
\par
\tabitem date: la fecha en la cual se hayan realizado mas revisiones.
\par
\tabitem size: el tamaño de revisión mas repetido.\\

\hline
\textbf{Descripción} & Adicionalmente, se pueden utilizar argumentos de filtrado, para especificar el conjunto de revisiones sobre el cuál se realiza el cálculo. Los argumentos disponibles hasta el momento son:
\par
\tabitem title: título del artículo extraído.
\par
\tabitem pageid: id del artículo extraído.
\par
\tabitem user: nombre del usuario que realiza la revisiones.
\par
\tabitem userid: id del usuario que realiza la revisiones.
\par
\tabitem tag: una etiqueta determinada que contenga las
revisiones.
\par
\tabitem size: el tamaño de la revisión realizada.
\par
\tabitem sizematch: valor que acompaña a \texttt{size}. Si el valor es
positivo, se filtrarán todas las revisiones de mayor
tamaño que el valor de \texttt{size}. Si es negativo, se filtrarán
todas las revisiones de menor tamaño que el valor de
\texttt{size}. Si el valor es 0 o no se encuentra en los parámetros
de la solicitud, se filtrarán todas las revisiones cuyo
tamaño sea exactamente el valor de \texttt{size}.
\par
\tabitem date: la fecha exacta en que fueron realizadas las
revisiones. El formato de fecha utilizado es: \texttt{YYYY-MM-DD}.
\par
\tabitem datestart: la fecha inicial a partir de la cual fueron
realizadas las revisiones en un intervalo de tiempo. El
formato de fecha utilizado es: \texttt{YYYY-MM-DD}. En caso de
que no exista el parámetro \texttt{dateend} en la solicitud, la fecha final del intervalo será la fecha actual.
\par
\tabitem dateend: la fecha final hasta la cual fueron realizadas las
revisiones en un intervalo de tiempo. El formato de
fecha utilizado es: \texttt{YYYY-MM-DD}. En caso de que no
exista el parámetro \texttt{datestart} en la solicitud, la fecha
inicial del intervalo será la fecha de la primera revisión
del artículo.
\\
\hline
\textbf{Observaciones} & Primero se realiza una comprobación de los parámetros de la
solicitud contra una lista blanca con los parámetros permitidos.
Luego, con los parámetros resultantes, se realiza el
procesamiento a través de una tarea asignada, para luego
devolver el resultado.

\par 
Se recomienda precaución al seleccionar los argumentos de filtrado, para evitar resultados redundantes. Por ejemplo, si se selecciona el atributo \texttt{user} como filtro y \texttt{attribute} tiene como valor \texttt{'user'}, el resultado será en el mejor caso el valor recibido por el filtro \texttt{user}, pues el conjunto seleccionado solo consistirá en revisiones realizadas por dicho usuario.
\\
\hline
\caption{Moda de Revisiones}
\label{tab:mode}
\end{longtable}
