\begin{longtable}{|l|m{4in}|}

\hline
\multicolumn{2}{|r|}{\textbf{Historia}} \\
\hline
\endfirsthead

\multicolumn{2}{c}%
{{\bfseries \tablename\ \thetable{} -- continuación de la página anterior}} \\
\hline \multicolumn{2}{|r|}{\textbf{Historia}} \\ \hline
\endhead

\textbf{Desarrollador} & Marvin Bernal \\
\hline
\textbf{Nombre} & Endpoint para el cálculo de la moda de las revisiones \\
\hline
\textbf{Sección} & Consulta\\
\hline
\textbf{Descripción} & Parte del API de la solución distribuida donde el endpoint \texttt{/count}
recibe como parámetros los atributos para restringir o filtrar el
rango a tomar en cuenta como resultado. Los argumentos
disponibles hasta el momento son:
\par
\tabitem title: título del artículo extraído.
\par
\tabitem pageid: id del artículo extraído.
\par
\tabitem user: nombre del usuario que realiza la revisiones.
\par
\tabitem userid: id del usuario que realiza la revisiones.
\par
\tabitem tag: una etiqueta determinada que contenga las
revisiones.
\par
\tabitem size: el tamaño de la revisión realizada.
\par
\tabitem sizematch: valor que acompaña a \texttt{size}. Si el valor es
positivo, se filtrarán todas las revisiones de mayor
tamaño que el valor de \texttt{size}. Si es negativo, se filtrarán
todas las revisiones de menor tamaño que el valor de
\texttt{size}. Si el valor es 0 o no se encuentra en los parámetros
de la solicitud, se filtrarán todas las revisiones cuyo
tamaño sea exactamente el valor de \texttt{size}.
\par
\tabitem date: la fecha exacta en que fueron realizadas las
revisiones. El formato de fecha utilizado es: \texttt{YYYY-MM-DD}.
\par
\tabitem datestart: la fecha inicial a partir de la cual fueron
realizadas las revisiones en un intervalo de tiempo. El
formato de fecha utilizado es: \texttt{YYYY-MM-DD}. En caso de
que no exista el parámetro \texttt{dateend} en la solicitud, la fecha final del intervalo será la fecha actual.
\par
\tabitem dateend: la fecha final hasta la cual fueron realizadas las
revisiones en un intervalo de tiempo. El formato de
fecha utilizado es: \texttt{YYYY-MM-DD}. En caso de que no
exista el parámetro \texttt{datestart} en la solicitud, la fecha
inicial del intervalo será la fecha de la primera revisión
del artículo.
\\
\hline
\textbf{Observaciones} & Primero se realiza una comprobación de los parámetros de la
solicitud contra una lista blanca con los parámetros permitidos.
Luego, con los parámetros resultantes, se realiza el
procesamiento a través de una tarea asignada, para luego
devolver el resultado.\\
\hline
\caption{Contabilización de Revisiones}
\label{tab:mode}
\end{longtable}
