\begin{longtable}{|l|m{4in}|}

\hline
\multicolumn{2}{|r|}{\textbf{Historia}} \\
\hline
\endfirsthead

\multicolumn{2}{c}%
{{\bfseries \tablename\ \thetable{} -- continuación de la página anterior}} \\
\hline \multicolumn{2}{|r|}{\textbf{Historia}} \\ \hline
\endhead

\textbf{Desarrollador} & Francisco Delgado \\
\hline
\textbf{Nombre} & Automatización de levantamiento de la aplicación \\
\hline
\textbf{Sección} & Configuración de aplicación distribuida \\
\hline
\textbf{Descripción} & Configuración de las imágenes y contenedores de docker para el levantamiento
de la aplicación, distribución de las tareas entre diversos nodos, y levantamiento del cluster de MongoDB.
\\
\hline
\textbf{Observaciones} & La configuración de los contenedores se realiza por medio del archivo \texttt{docker-compose.yml}. Se implementó un archivo por cada ambiente de desarrollo, los cuales son: Digital Ocean, el cual es un servicio de alojamiento web y sirve como ambiente de producción sin la inclusion de las réplicas de mongo;
Desarrollo, para el ambiente para desarrollo continuo de la aplicación;
y por ultimo, Réplica, el cual define la configuración completa del cluster de MongoDB y sirve como ambiente de producción \\
\hline
\caption{Configuración del levantamiento de la aplicación por medio de Docker}
\label{tab:docker}
\end{longtable}
