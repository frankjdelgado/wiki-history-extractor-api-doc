\begin{longtable}{|l|m{4in}|}

\hline
\multicolumn{2}{|r|}{\textbf{Historia}} \\
\hline
\endfirsthead

\multicolumn{2}{c}%
{{\bfseries \tablename\ \thetable{} -- continuación de la página anterior}} \\
\hline \multicolumn{2}{|r|}{\textbf{Historia}} \\ \hline
\endhead

\textbf{Desarrollador} & Francisco Delgado \\
\hline
\textbf{Nombre} & Configuración de Shards y Replicas  \\
\hline
\textbf{Sección} & Base de Datos \\
\hline
\textbf{Descripción} & Configuración de los diversos nodos que incluidos dentro del cluster
de la base de datos.
Se configuran dos servidores para la fragmentación de datos por medio de Hashed Sharding.
Cada un de ellos forma parte de un grupo de replicas, por lo que se hace uso de tres servidores
por nodo de fragmentación.
\par
Se elige un nodo principal por cada grupo de replica, en este nodo se inicializa la configuración del grupo por medio del comando \verb|rs.initiate()|.
Posteriormente se agregan los dos miembros restantes del grupo usando \verb|rs.add('mongors1n2:27018')| y \verb|rs.add('mongors1n3:27018')|.
Luego, con estos grupos de replicas ya creados, se generan los shards con el siguiente comando:
\verb|sh.addShard('mongors1/mongors1n1:27018')|. Y, finalmente, se habilita el sharding por medio del comando \verb|sh.enableSharding('wiki_history_extractor')|

Una vez las replicas y shards, han sido configurados, se genera una replica extra para el servidor de configuración, el cual ayuda a la coordinación y consulta de los shards.

\\
\hline
\textbf{Observaciones} & Para los grupos de replicas no fue implementado un algoritmo de selección
de nodo maestro, por lo cual se hace uso del comportamiento por defecto de
MongoDB para elegirlo.\\
\hline
\caption{Configuración del cluster de MongoDB}
\label{tab:cluster}
\end{longtable}
