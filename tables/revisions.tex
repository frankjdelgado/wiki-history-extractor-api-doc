\begin{longtable}{|l|m{4in}|}

\hline
\multicolumn{2}{|r|}{\textbf{Historia}} \\
\hline
\endfirsthead

\multicolumn{2}{c}%
{{\bfseries \tablename\ \thetable{} -- continuación de la página anterior}} \\
\hline \multicolumn{2}{|r|}{\textbf{Historia}} \\ \hline
\endhead

\textbf{Desarrollador} & Marvin Bernal \\
\hline
\textbf{Nombre} & Endpoint para consulta de revisiones \\
\hline
\textbf{Sección} & Consulta\\
\hline
\textbf{Descripción} & Parte del API de la solución distribuida donde el endpoint \texttt{/revisions}
recibe como parámetros los atributos para restringir o filtrar el
rango a tomar en cuenta como resultado. Los argumentos
disponibles hasta el momento son:
\par
\tabitem comment: comentario de la revisión.
\par
\tabitem anon: la revisión fue fecha por un autor anónimo.
\par
\tabitem pageid: id del artículo extraído.
\par
\tabitem user: nombre del usuario que realiza la revisiones.
\par
\tabitem userid: id del usuario que realiza la revisiones.
\par
\tabitem tags: etiqueta o etiquetas que contengan las
revisiones.
\par
\tabitem revid: id de la revisión.
\par
\tabitem contentformat: el formato del contenido.
\par
\tabitem contentmodel: el modelo del contenido.
\par
\tabitem extraction\_date: la fecha en que se extrajo la revisión.
\par
\tabitem parentid: id de la revisión padre.
\par
\tabitem title: título del artículo wiki.
\par
\tabitem minor: la revisión fue una edición menor.
\par
\tabitem size: el tamaño de la revisión realizada.

\par Adicionalmente la consulta se realiza de forma paginada y puede ser ordenada por fechas. 
Se puede manejar la paginación y el ordenamiento con los siguiente argumentos:
\par
\tabitem page\_size: cantidad de revisiones por página.
El valor por defecto es 20, máximo valor es 200.
\par
\tabitem page: página de revisiones a mostrar.
El valor por defecto es 1.
\par
\tabitem sort: se indica si las revisiones se muestran en orden descendente o ascendente. Los posibles valores son \texttt{asc} o \texttt{desc}
\\
\hline
\textbf{Observaciones} & \\
\hline
\caption{Consulta de Revisiones}
\label{tab:revisions}
\end{longtable}
