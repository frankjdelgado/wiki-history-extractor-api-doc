\section*{introducción}

Hoy en día Wikipedia, la enciclopedia libre, se ha posicionado como una de las herramientas de
referencia más grandes e indispensables para todo aquel en búsqueda de información.

Wikipedia cuenta con 30 millones de usuarios registrados, de los cuales mas de 130 mil son
usuarios activos que contribuyen en hasta 270 idiomas generando nuevo contenido, perfeccionándolo,
removiendo errores e incluso vandalismos en los artículos \cite{1}.

Cada uno de estos cambios son almacenados en un historial en conjunto con una serie de datos
adicionales tales como el autor de la actualización, la fecha, notas, entre otras. Existen
aplicaciones, tales como Wiki-Metrics-UCV, que tienen como objetivo la recolección, almacenamiento
y procesamiento de historiales para el cálculo y visualización de métricas de sus atributos \cite{4}.

Para la fecha, Wikipedia alberga alrededor de 5.5 millones de artículos, y que
por cada uno de ellos existe un promedio de 20 mil actualizaciones \cite{2}. Las aplicaciones de terceros,
como Wiki-Metrics-UCV, que están enfocadas en la extracción de estos y son desarrolladas
en un esquema centralizado, no tienen la capacidad de escalar sus servicios una vez alcancen
los limites de sus unidades de almacenamiento.

Por suerte, existe el modelo de cómputo distribuido en el cual la carga de datos y las tareas llevadas a
cabo en un sistema de software son compartidas por múltiples componentes para aumentar la
eficiencia y el rendimiento.

La aplicación de los sistemas distribuidos se adapta a diversos campos, tales como:
redes de telecomunicaciones, computación paralela y procesos de monitoreo en tiempo
real, entre otros. El uso de este tipo de sistemas puede llegar a ser muy beneficioso por
razones de rendimiento o de costo, por ejemplo, puede ser menos costoso el uso de un grupo de varios
computadores con componentes de gama baja, que un solo computador con componentes gama alta. Ademas,
un sistema distribuido puede llegar proporcionar más fiabilidad que un sistema centralizado, ya
que pueden implementarse medidas de recuperación de fallos en caso de que uno de los miembros
del sistema presente un error.

Con este tipo modelo es posible diseñar un sistema en donde todos sus componentes se dividan
de forma equitativa las tareas necesarias para resolver un problema común, tal como el almacenamiento y
procesamiento de miles de historiales de los artículos de Wikipedia. De esta manera cada componente podría
almacenar los datos recolectados en su propia unidad de almacenamiento y así garantizar que el sistema
sea escalable.

Este trabajo documenta el desarrollo de un sistema distribuido, basado en Wiki-Metrics-UCV,
para la extracción, almacenamiento y procesamiento del historial de artículos wikis basados en MediaWiki (por ejemplo, Wikipedia).

En el Capítulo 1 se presenta en detalle la descripción del problema, la justificación del por qué
desarrollar una alternativa distribuida al mismo, y por último, los objetivos tantos generales como
específicos que se llevaron a cabo.

En el Capítulo 2 se describen tecnologías utilizadas en el desarrollo del sistema, incluyendo servicios,
las librerías usadas y las bases de datos elegidas.

Por último, en el Capítulo 3 se presenta el marco aplicativo del sistema desarrollado para la solución al problema planteado,
en donde se describen sus componentes y el método de desarrollo.
