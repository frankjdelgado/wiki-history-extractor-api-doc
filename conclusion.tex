\section{Conclusiones}

El análisis de los historiales de los artículos wiki basados en Wikimedia permite
descubrir diversos indicadores y características que no pueden ser detalladas a simple
vista.
Sin embargo, para llevar a cabo estos análisis, es necesario almacenarlos en una base
de datos, los cuales, debido a la gran cantidad de datos que pueda envolver, resulta
en una tarea complicada de ejecutar debido a las limitaciones de hardware de la
máquina.
Estas limitaciones pueden variar desde el espacio de almacenamiento hasta disponibilidad
de servicios y recuperación de fallos, para lo casos de las aplicaciones o servicios como
Wiki-Metrics-UCV.
Hoy en día, la necesidad de diseñar aplicaciones con una arquitectura distribuida
ha incrementado, puesto que, en muchos casos, facilita la resolución de estos problemas
a menor costo económico que, por ejemplo, mejorar la capacidad individual de una sola máquina.

Aunque la implementación de una arquitectura distribuida puede ser complicada,
hoy en día existen herramientas facilitan esta tarea en gran medida.
Tal es el caso del sistema de administración de bases de datos MongoDB, cuya gestión
de datos permite implementar, de manera sencilla, aspectos como la replicación de
de datos, que influye directamente en la habilidad de prestar un servicio de alta disponibilidad,
y la fragmentación de la base de datos entre múltiples máquinas.

El uso de MongoDB, en conjunto con herramientas como Celery y RabbitMQ, proporcionan
una gran ayuda a la hora de mitigar las limitaciones que conlleva el uso de un sistema centralizado, y el cumplimiento de los objetivos planteados para este Trabajo Especial de Grado.

Por medio de Celery es posible ejecutar múltiples tareas, cómo la extracción de métricas de los
datos almacenados, evitando, por ejemplo, el colapso de un servicio web, e incluso, brinda
de manera moderada la habilidad de recuperarse de fallos en tiempos de ejecución, puesto que gracias a RabbitMQ,
es posible llevar un registro de su progreso y de las tareas pendientes a ejecutar.

El desarrollo de un nuevo algoritmo para la actualización automática de las revisiones ya almacenadas, toma en cuenta nuevos parámetros que permiten una evaluación más adecuada de cada artículo.
De esta forma se redujo la cantidad de peticiones innecesarias sobre el API de MediaWiki, incrementando la eficiencia del sistema.

La inclusión de Flask, para el desarrollo del API, permite a las aplicaciones de terceros consultar y extraer
los historiales del artículo wiki deseado, además de brindar las herramientas necesarias la ejecución
de consultas más especializadas y la visualización de métricas previamente definidas ya incluidas en el sistema.

La arquitectura del sistema distribuido, descrito en este documento, brinda
una solución escalable a los problemas previamente planteados, en donde la capacidad
de un sistema centralizado es insuficiente ante la demanda progresiva de las aplicaciones de
extracción, consulta y procesamiento de historiales de artículos wiki.

En caso de tener una gran cantidad de revisiones que extraer, los lineamientos de uso del API de MediaWiki limitan el uso de múltiples nodos para agilizar el proceso.
A pesar de que existen múltiples nodos de extracción trabajando de forma paralela, puede que la duración del proceso de extracción se extienda demasiado, dado que existe 1 segundo de intervalo por cada página de revisiones.

Por otro lado, la ejecución de las tareas de Celery depende del servicio que provee RabbitMQ, el cual es centralizado.
Por lo tanto, en caso de que este servicio presente fallas, el resto de los nodos involucrados en el sistema se paraliza.

Con respecto a la base de datos, dado que la misma se encuentra particionada entre diversos nodos separados geográficamente, el tiempo de respuesta para cada consulta puede ser elevado.
Lo mismo se aplica para los diversos nodos que ejecutan el API, su ubicación en referencia al origen de las peticiones de consultas, afecta el tiempo de respuesta

Tomando en cuenta estas limitaciones, se propone implementar nuevos algoritmos para la partición de datos entre los nodos que conforman las particiones de MongoDB, y así mejorar su distribución.

Adicionalmente, se propone la inclusión de las técnicas y diseño de bases de datos cache para mejorar la escalabilidad con nuevos nodos de menor capacidad de almacenamiento que atiendan la consulta de los datos más usados, y además, mejorar su disponibilidad y la velocidad con la que son servidos.

Finalmente, se recomienda la descentralización de RabbitMQ a través de un escalamiento horizontal que permita la tolerancia a fallos, y a su vez, incrementar la disponibilidad del servicio.
